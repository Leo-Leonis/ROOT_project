Nella seguente relazione si intende simulare eventi di collisioni tra vari tipi di particelle e fare analisi di dati a riguardo. Per la realizzazione del programma si è utilizzato il linguaggio C++ con la libreria ROOT, con particolare enfasi sulla programmazione ad oggetti, sulla simulazione di eventi fisici tramite generazione Monte Carlo e infine sull'analisi con ROOT.\\

I tipi di particelle coinvolte posso essere raggruppati in due categorie: quelli stabili e quelli molto instabili, ossia risonanze. Per lo scopo di questo progetto, di queste particelle andiamo a considerare solo queste caratteristiche: massa, carica, quantità di moto. Solo alle risonanze però si associa un'altra caratteristica, la larghezza di risonanza. Inoltre le particelle considerate sono: pioni ($\pi^\pm$), kaoni (K$^\pm$), protoni (p$^\pm$) e infine K$^*$, con K$^*$ la particella di risonanza. \\

Nel corso delle interazioni tra queste particelle, emergeranno i parametri di K$^*$, e lo scopo di questo progetto è riuscire a determinare questi parametri.\\